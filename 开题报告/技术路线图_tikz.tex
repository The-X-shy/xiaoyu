\documentclass[12pt,a4paper,fontset=fandol]{ctexart}
\usepackage{tikz}
\usetikzlibrary{arrows.meta,positioning,calc,fit,backgrounds,shadows}
\usepackage[active,tightpage]{preview}
\PreviewEnvironment{tikzpicture}
\setlength\PreviewBorder{8pt}
\pagestyle{empty}

\begin{document}
\begin{tikzpicture}[node distance=1.95cm, every node/.style={font=\small}]

% --- 颜色定义 ---
\definecolor{mainblue}{RGB}{36,96,173}
\definecolor{lightblue}{RGB}{238,245,255}
\definecolor{keyorange}{RGB}{255,247,235}
\definecolor{phasebg}{RGB}{225,237,255}

% --- 样式 ---
\tikzset{
  process/.style={
    rectangle,
    minimum width=5.2cm,
    minimum height=1.25cm,
    text centered,
    text width=4.8cm,
    rounded corners=4pt,
    draw=mainblue!85,
    line width=0.9pt,
    fill=lightblue,
    drop shadow={shadow xshift=1.2pt,shadow yshift=-1.2pt,opacity=0.18}
  },
  keyprocess/.style={
    process,
    draw=orange!70!black,
    fill=keyorange,
    line width=1.05pt
  },
  startbox/.style={
    process,
    fill=white,
    draw=mainblue,
    line width=1.25pt
  },
  phase/.style={
    rectangle,
    rounded corners=8pt,
    minimum width=2.7cm,
    minimum height=0.72cm,
    text centered,
    fill=phasebg,
    draw=mainblue!35,
    line width=0.6pt,
    font=\bfseries\small,
    text=mainblue!85!black
  },
  arrow/.style={
    -{Latex[length=2.8mm,width=2.1mm]},
    line width=1.05pt,
    draw=mainblue!85
  }
}

% --- 节点 ---
\node (start) [startbox] {
  \textbf{课题:基于自适应光斑匹配(ASM)的}\\[-1pt]
  \textbf{SHWS动态范围软件扩展方法研究}
};

\node (sim) [process, below=0.68cm of start] {
  \textbf{SHWS 全链路数值仿真环境构建}\\[-1pt]
  \footnotesize 波前生成 $\rightarrow$ 微透镜采样 $\rightarrow$ 焦斑成像 $\rightarrow$ 探测器噪声建模
};

\node (baseline) [process, below left=1.28cm and 3.75cm of sim] {
  \textbf{对照组:基线方法实现}\\[-1pt]
  \footnotesize 传统质心探测 + 迭代外推策略 (Extrapolation)
};

\node (asm) [keyprocess, below right=1.28cm and 3.75cm of sim] {
  \textbf{实验组:ASM 核心算法设计}\\[-1pt]
  \footnotesize Hausdorff 距离目标函数构建 + 粒子群优化 (PSO) 全局搜索
};

\node (eval) [process, below=2.45cm of sim] {
  \textbf{主实验:等条件对比分析}\\[-1pt]
  \footnotesize 统一数据集与噪声水平下的:动态范围极限与波前重建精度 (RMSE)
};

\node (model) [process, below=0.68cm of eval] {
  \textbf{参数敏感度与量化建模}\\[-1pt]
  \footnotesize 微透镜参数扫描 ($f$, $d$) $\rightarrow$ 建立范围-参数关系模型 $R=F(\cdot)$
};

\node (end) [process, below=0.68cm of model, fill=white] {
  \textbf{总结展望与论文撰写}\\[-1pt]
  \footnotesize 提炼工程设计准则
};

% --- 连线 ---
\draw[arrow] (start) -- (sim);
\draw[arrow] (sim) -- ++(0,-0.72) -| (baseline);
\draw[arrow] (sim) -- ++(0,-0.72) -| (asm);
\draw[arrow] (baseline) |- (eval);
\draw[arrow] (asm) |- (eval);
\draw[arrow] (eval) -- (model);
\draw[arrow] (model) -- (end);



% --- 分组框 ---
\begin{scope}[on background layer]
  \node[
    rounded corners=7pt,
    draw=orange!70!black,
    dashed,
    line width=0.8pt,
    inner sep=10pt,
    fit=(baseline) (asm)
  ] (algbox) {};
\end{scope}

\node[
  fill=white,
  text=orange!60!black,
  font=\bfseries\footnotesize,
  inner sep=1.5pt
] at ($(algbox.north)+(0,0.28)$) {核心算法实现层};

\end{tikzpicture}
\end{document}
