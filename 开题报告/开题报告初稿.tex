\documentclass[12pt,a4paper,fontset=fandol]{ctexart}
\usepackage[margin=2.5cm]{geometry}
\usepackage{setspace}
\usepackage{booktabs}
\usepackage{hyperref}
\usepackage{indentfirst}
\usepackage{graphicx}
\usepackage{float}
\hypersetup{colorlinks=true,linkcolor=black,citecolor=black,urlcolor=blue}
\setstretch{1.3}
\setlength{\parindent}{2em}
\setlength{\parskip}{0pt}

\title{本科毕业设计开题报告(初稿)\\
\large 基于自适应光斑匹配的夏克--哈特曼波前传感器动态范围软件扩展研究}
\author{学生姓名:\underline{\hspace{3cm}} \quad 学号:\underline{\hspace{3cm}}\\
专业:\underline{\hspace{3cm}} \quad 指导教师:\underline{\hspace{3cm}}}
\date{\today}

\begin{document}
\maketitle

\section{选题依据}
\noindent(简述该选题的研究意义和背景,国内外研究概况和发展趋势等)

\subsection{选题背景及研究意义}
夏克--哈特曼波前传感器(Shack-Hartmann Wavefront Sensor,SHWS)是自适应光学、激光光束诊断、眼科像差检测和精密光学测量中的常用波前探测手段。其基本过程是通过微透镜阵列对入射波前进行子孔径采样,根据焦斑偏移估计局部斜率,再由斜率场完成波前重构。该方法结构紧凑、响应速度快、工程实现成熟,因此在需要稳定波前反馈的光电系统中应用广泛。

随着应用场景向大像差和强扰动工况发展,SHWS动态范围不足的问题逐步凸显。当局部斜率较大时,焦斑可能跨越子孔径边界或偏离探测区域,传统子孔径内最近邻匹配容易失效,匹配误差会沿重构链路传播并造成系统性失真。这类误差并非随机波动,而是对应关系错误引起的结构性偏差,会直接影响闭环控制的稳定性和测量结果的可靠性。

从系统设计角度看,灵敏度与动态范围之间长期存在耦合权衡。缩短焦距可提高可容许位移,但会降低斜率分辨率;增大焦距可以提高灵敏度,却会增加跨孔径失配风险。硬件改造可以在一定程度上缓解矛盾,但通常伴随系统复杂度、标定成本和维护压力的上升。相比之下,软件扩展方法在不改变主光路的前提下提升可测范围,具有更好的部署灵活性和工程可行性。

本课题以软件扩展路线为主线,聚焦自适应光斑匹配(Adaptive Spot Matching,ASM)方法,面向机理解释、方法比较和参数设计三个层面开展研究。课题希望在统一仿真环境下解释动态范围受限机理,客观比较基线方法与ASM方法的性能差异,并形成可用于参数选型的定量结论。

\subsection{国内外研究现状和发展趋势}
现有文献主要沿两条路线展开。一条路线是排序、外推、邻域连续性等局部启发式方法,另一条路线是将光斑对应关系转化为全局优化问题。两类方法在鲁棒性、失效形态和可解释性方面表现不同,在强像差和缺斑场景中的差异尤为明显。

\subsubsection{SHWS动态范围扩展研究进展}
早期研究给出了可测斜率的几何边界,并提出多种局部匹配策略。排序法和外推法在中等像差条件下能够在不改动硬件的情况下提升实用动态范围,算法实现相对简洁、计算代价较低,至今仍是重要对照方法。

后续研究发现,当焦斑拓扑关系被大位移、噪声或缺斑破坏时,局部传播策略会出现误差累积。一旦关键节点匹配错误,后续区域的对应关系会持续偏离真实情况,最终造成重构失稳。该现象推动研究从局部递推转向全局一致性约束。

\subsubsection{ASM方法及发展趋势}
近年来,ASM作为全局优化路线受到关注。该方法建立“Zernike系数到光斑点集”的前向映射,在参数空间中搜索与观测点集最匹配的估计结果。其目标函数采用Hausdorff距离衡量集合级一致性,并通过惩罚项抑制多对一退化匹配,再利用粒子群优化(Particle Swarm Optimization,PSO)完成非凸搜索。

已有文献表明,在典型测试条件下,ASM动态范围可提升至14倍以上,并在部分光斑缺失场景下保持可用重构精度。该结果表明,ASM不仅能够扩展可测范围,也能在临界工况附近保持较稳定的重构表现。

从研究趋势看,动态范围评估正在从单次验证转向可复现、可比较的统一实验范式。统一样本、统一噪声模型和统一误差阈值正在成为常见要求。深度学习方法也被用于波前反演,具备一定潜力,但其性能依赖训练域覆盖和迁移一致性。基于课题目标与可解释性要求,本研究将深度学习仅作为背景对照,不纳入核心实现路线。

\section{研究目标和内容}
\subsection{研究目标}
本课题旨在建立可复现的SHWS动态范围软件扩展流程,并形成可用于设计分析的参数关系模型。具体而言,课题将构建完整仿真平台,实现基线方法与ASM方法在统一条件下的对比,采用动态范围与重构精度双约束评估方法有效性,并给出可操作的适用边界与参数建议。

\subsection{研究内容}
研究首先搭建SHWS前向仿真链路,包含波前生成、子孔径采样、焦斑成像、探测器采样和噪声注入。随后实现“外推+最近邻”基线流程,作为统一对照。基于同一仿真环境实现ASM算法,构建Hausdorff距离目标函数、失配惩罚项和PSO搜索流程。

在算法实现基础上,研究将对焦距、间距、有效口径、像元尺寸和噪声水平进行参数扫描,建立$R=F(p,f,d,s,\sigma)$离散关系模型,其中$R$表示误差约束下的可用动态范围。最终通过范围增益、重构误差、缺斑鲁棒性和计算代价等指标进行综合比较,形成可复核结论。

本课题总体技术路线如图所示。

\begin{figure}[H]
  \centering
  \includegraphics[width=0.88\textwidth]{技术路线图_tikz.pdf}
  \caption{总体技术路线}
  \label{fig:tech-route}
\end{figure}

\subsection{关键问题}
本课题围绕三个问题展开:其一,焦斑跨孔径与缺斑并存时如何保持匹配稳定;其二,如何构建可辨识且数值稳定的ASM目标函数;其三,如何在扩展动态范围的同时维持可接受的重构精度,避免出现范围提升但结果不可用的情况。

\section{研究方案}
\subsection{总体技术路线}
本研究采用“机理分析--统一建模--算法对比--参数建模--工程解释”的闭环路线。首先构建统一仿真环境,保证算法差异不被数据生成差异掩盖;随后在相同样本与相同重构框架下比较基线方法与ASM方法;最后通过参数敏感性分析和边界识别把实验结果转化为设计建议。

\subsection{SHWS基础仿真模型}
仿真模型包含波前生成、斜率提取、焦斑形成、探测器采样、质心估计、匹配与重构模块。波前样本覆盖离焦、像散、彗差及混合Zernike组合。噪声模型包括读出噪声、背景噪声和量化误差。模型有效性通过两类检验确认:一是小斜率下的几何一致性检验,二是噪声增量条件下的误差传播趋势检验。

\subsection{ASM算法实现}
基线组采用外推引导的最近邻匹配,作为局部逻辑方法参照。ASM组以候选Zernike参数生成估计点集,通过Hausdorff距离计算集合差异,并加入惩罚项抑制不合理对应关系。PSO迭代更新候选解并输出最优参数。比较指标包括匹配成功率、重构RMSE、最大可用斜率和计算耗时。

\subsection{参数关系建模与定量分析}
动态范围定义为在预设误差阈值内仍可稳定重构的最大斜率幅值,同时以RMSE作为精度约束。分析流程包含单因素趋势分析和双因素耦合分析,重点识别参数变化方向及交互效应。研究不追求单点最优,而是识别可用参数区间,以提高结论在工程场景中的可迁移性。

\subsection{实验设计与评价方法}
所有对比实验共享相同数据生成流程和重构流程,测试集覆盖逐级增强像差并包含缺斑场景。评价维度包括动态范围极限、重构精度、鲁棒性与计算效率。结果通过趋势曲线、参数热图和临界样本重构结果联合呈现,以减少单一指标导致的偏差。

\subsection{风险控制与结果交付}
潜在风险主要包括模型失配、临界工况下优化不稳定和大规模扫描带来的计算开销。应对策略包括分阶段校核、困难样本的保守回退机制以及粗到细的扫描计划。最终交付内容包括可复现实验代码、固定随机种子的实验记录、关键图表和参数推荐表,并形成可直接用于论文写作的证据链。

\section{研究计划及进度安排}
研究进度按“准备--实现--验证--成文”推进。第1至2周完成文献梳理与指标口径定义。第3至4周完成基础仿真模块和基线流程校核。第5至6周完成ASM目标函数与PSO求解模块。第7至8周开展核心对比实验。第9至11周完成缺斑与噪声鲁棒性测试。第12至14周完成参数扫描和关系建模。第15至16周完成结果整合、论文写作和答辩准备。

\section{创新点及预期研究成果}
\subsection{创新点}
本课题的创新主要体现在三个方面。第一,在可复现实验框架中实现ASM流程,把Hausdorff集合匹配与PSO搜索结合到同一工程链路。第二,在误差约束下建立动态范围与微透镜参数的定量关系,提升结论的设计指导价值。第三,构建统一对比协议,强调范围增益、精度保持与边界稳定性的联合评估。

\subsection{预期研究成果}
课题预期形成完整的Python仿真与评估工具链,获得基线与ASM的系统对比数据,给出参数推荐区间及适用边界,并完成毕业论文与答辩材料。相关流程可直接复用于后续研究,以支持更透明的算法验证和参数化系统调优。

\section{六、参考文献}
\begin{enumerate}
  \item He Q, Wang Z, et al. Large dynamic range Shack-Hartmann wavefront sensor based on adaptive spot matching[J]. Light: Advanced Manufacturing, 2024.
  \item Lee J, Shack R V, Descour M R. Sorting method to extend the dynamic range of the Shack-Hartmann wave-front sensor[J]. Applied Optics, 2005, 44(23): 4838-4845.
  \item Podanchuk D V, Dan'ko V P, Son J Y. Extended-range Shack-Hartmann wavefront sensor[J]. Optical Engineering, 2006, 45(5): 053605.
  \item 李华强,宋贺伦,饶长辉,等.增大夏克-哈特曼波前传感器动态测量范围的方法[J].光学精密工程,2008,16(7):1203-1207.
  \item 张金平,张忠玉,张学军,等.增大夏克-哈特曼波前传感器动态范围的算法研究[J].光学学报,2011,31(8):0812006.
  \item 韩妍娜,胡新奇,董冰.一种扩大夏克-哈特曼波前传感器动态范围的迭代外推法[J].光学学报,2020,40(16):1611004.
\end{enumerate}

\end{document}
